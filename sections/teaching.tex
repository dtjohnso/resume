\roottitle{Teaching Experience}

\headedsection
{\href{http://www.foundationbaptistcollege.com/}{Foundation Baptist College}}
{\textsc{Edmonton, \acr{AB}}} 
{
	\headedsubsection{AL401 Matthew in Greek}{Fall 2013}
		{\bodytext{ Translation and exegesis of the first Gospel, with studies in syntax and vocabulary.}}
	\headedsubsection{Ex201 Exposition of John's Gospel}{Fall 2012, Spring 2015}
		{\bodytext{ Analyzes major theological themes and exposits select portions in John. Highlights hermeneutical principles vital to a proper exposition of John's writings.}}
	\headedsubsection{Ex204 Exposition of Isaiah}{Summer 2012, Spring 2015} 
		{\bodytext{An exposition of the book of Isaiah emphasizing the book's unified structure and thematic development. Includes discussion of how to structure a preaching series from Isaiah and vital hermeneutical principles within the book.}}
	\headedsubsection{Ex315 Exposition of John's Epistles}{Spring 2014} 
		{\bodytext{An exposition of 1, 2, and 3 John, with an emphasis on John's theology.}}
	\headedsubsection{Ex401 Exposition of the Psalms}{Fall 2015}
		{\bodytext{Exposits select psalms representative of each genre and examines Hebrew poetry and the historical background of the Psalms.}}
	\headedsubsection{Ex402 Exposition of Mark}{Spring 2013, Fall 2015} 
		{\bodytext{An exposition of the gospel of Mark. Emphasizes its role in the synoptic gospels and its theme. Discusses how to interpret New Testament narrative as well as parabolic literature and discourse.}}
	\headedsubsection{Pr101 Preparing Bible Messages}{Fall 2015, Fall 2016}
		{\bodytext{Discusses the methodology for preparing a message from a biblical text. Defines exposition and details the steps of explanation and application.}}
	\headedsubsection{Pr103 Personal Evangelism}{Spring 2012, Fall 2014, Spring 2015, Spring 2016, Spring 2017}
		{\bodytext{Establishes the believer's obligation to evangelize. Demonstrates both the theology and methodologies for carrying out this obligation.}}
	% \headedsubsection{Pr204 Hermeneutics}{Fall 2016}
	% 	{\bodytext{Focuses on principles of hermeneutics. Equips the student to accurately explain and apply the Scripture.}}
	\headedsubsection{Su101 Old Testament Messages}{Fall 2015, Fall 2016}
		{\bodytext{A study of the unfolding theme of the Bible as it is revealed through the messages of the Pentateuch and the history, poetry and prophecy books of the Old Testament.}}
	\headedsubsection{Su102 New Testament Messages}{Spring 2016, Spring 2017}
		{\bodytext{The unfolding theme of the Bible as it is revealed through the messages of the New Testament books, with discussion of key passages within each New Testament book.}}
	\headedsubsection{Su201 Life of Christ}{Fall 2011, Fall 2014}
		{\bodytext{A survey of the life of Christ as recorded in the four canonical gospels, with an emphasis on the application of a traditional grammatical-historical hermeneutic to the biblical text.}}
	\headedsubsection{Su401 Historical Books}{Spring 2014}
		{\bodytext{ Overviews Joshua through Esther, emphasizing chronology and historical background. Highlights the introduction to each book (date of composition, author, occasion \& purpose, and recipients) and themes. Discusses contemporary application.}}
	\headedsubsection{Th101 Systematic Theology I}{Fall 2015, Fall 2016}
		{\bodytext{Introduces the student to the two disciplines of theology (biblical \& systematic), and covers Bibliology, Theology Proper, Angelology, Anthropology, and Christology.}}
	\headedsubsection{Th104 Bibliology}{Spring 2016, Spring 2017}
		{\bodytext{Defines the nature of Scripture as divine revelation and describes the history of the sacred text. Explains the theological and practical implications of the Bible's inspiration and inerrancy while also summarizing the history of canonization, manuscript transmission, and translation.}}
	\headedsubsection{Th105 Church History Survey}{Spring 2014, Summer 2014, Fall 2015, Fall 2016} 
		{\bodytext{Surveys the people, places and dates most important in understanding how the Holy Spirit has been saving, sanctifying and organizing people for the past two millennia.}}
	\headedsubsection{Th106 Baptist History}{Fall 2013, Spring 2016, Spring 2017}
		{\bodytext{Traces the history of the present Baptist movement back to 17th century England. Examines the three primary theories of Baptist history and traces the spread of Baptist teaching throughout England, North America and the world.}}
	\headedsubsection{Th313 Reformation}{Spring 2016}
		{\bodytext{Focuses on Luther, Calvin and Tyndale with the background of the pre-reformation period. Examines the contribution of each reformer to the doctrine and prosperity of the Reformation.}}

}

\headedsection
{\href{http://www.bju.edu/}{Bob Jones University}}
{\textsc{Greenville, \acr{SC}, \acr{USA}}} 
{
	
	{\bodytext{I developed a series of workshops titled the \emph{Seminary Project Labs}, and delivered these several times in 2010 and 2011. All of the sessions were webcasted, and archived footage from the session webcasts is available on \href{http://libguides.bju.edu/seminary}{the workshop website}.}}
	
	\headedsubsection{Greek and Hebrew Fonts}{}
		{\bodytext{Compares and contrasts the use of typefaces which display English characters as Greek/Hebrew glyphs with typefaces which support a full Unicode character set. Encourages users to adopt Unicode, and demonstrates how to configure Unicode support on the Windows operating system (some resources for Mac OS also available).}}
	\headedsubsection{Turabian}{}
		{\bodytext{Provides a general orientation to the \textit{Manual for Writers} written by Kate Turabian, \textit{et al.} Demonstrates the use of templates in Word and OpenOffice to properly lay out a Turabian style document.}}
	\headedsubsection{Zotero}{}
		{\bodytext{Demonstrates the use of Zotero, a powerful citation management tool that enables writers to track information found in research sources and properly cite those sources in an academic paper.}}

}